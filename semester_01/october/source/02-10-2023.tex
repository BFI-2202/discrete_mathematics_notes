\documentclass{article}
\usepackage[utf8]{inputenc}

\usepackage[T2A]{fontenc}
\usepackage[utf8]{inputenc}
\usepackage[russian]{babel}

\usepackage{tabularx}
\usepackage{amsmath}
\usepackage{pgfplots}
\usepackage{geometry}
\geometry{
    left=1cm,right=1cm,top=2cm,bottom=2cm
}
\newcommand*\diff{\mathop{}\!\mathrm{d}}

\newtheorem{definition}{Определение}
\newtheorem{theorem}{Теорема}

\DeclareMathOperator{\sign}{sign}

\usepackage{hyperref}
\hypersetup{
    colorlinks, citecolor=black, filecolor=black, linkcolor=black, urlcolor=black
}

\title{Дискретная математика}
\author{Лисид Лаконский}
\date{October 2023}

\begin{document}
\raggedright

\maketitle

\tableofcontents
\pagebreak

\section{Практическое занятие — 02.10.2023}

\subsection{Решение булевых систем уравнений}

\paragraph{Пример №1}

\begin{equation}
    \begin{cases}
        x_1 \oplus x_2 \oplus x_5 = 0 \\
        x_2 \oplus x_3 \oplus x_4 = 1 \\
        x_1 \oplus x_3 \oplus x_5 = 0 \\
        x_2 \oplus x_3 \oplus x_4 \oplus x_5 = 0 \\
        x_1 \oplus x_4 = 0
    \end{cases}
\end{equation}

\textbf{Матричный метод решения}:

$$
\begin{pmatrix}
    1 & 1 & 0 & 0 & 1 & 0 \\
    0 & 1 & 1 & 1 & 0 & 1 \\
    1 & 0 & 1 & 0 & 1 & 0 \\
    0 & 1 & 1 & 1 & 1 & 0 \\
    1 & 0 & 0 & 1 & 0 & 0
\end{pmatrix}
$$

Преобразуем данную матрицу к ступенчатому виду:

$$
\begin{pmatrix}
    1 & 1 & 0 & 0 & 1 & 0 \\
    0 & 1 & 1 & 1 & 0 & 1 \\
    1 & 0 & 1 & 0 & 1 & 0 \\
    0 & 1 & 1 & 1 & 1 & 0 \\
    1 & 0 & 0 & 1 & 0 & 0
\end{pmatrix} \sim
\begin{pmatrix}
    1 & 1 & 0 & 0 & 1 & 0 \\
    0 & 1 & 1 & 1 & 0 & 1 \\
    0 & 1 & 1 & 0 & 0 & 0 \\
    0 & 1 & 1 & 1 & 1 & 0 \\
    0 & 1 & 0 & 1 & 1 & 0
\end{pmatrix} \sim
\begin{pmatrix}
    1 & 1 & 0 & 0 & 1 & 0 \\
    0 & 1 & 1 & 1 & 0 & 1 \\
    0 & 0 & 0 & 1 & 0 & 1 \\
    0 & 0 & 0 & 0 & 1 & 1 \\
    0 & 0 & 1 & 0 & 1 & 1
\end{pmatrix} \sim
\begin{pmatrix}
    1 & 1 & 0 & 0 & 1 & 0 \\
    0 & 1 & 1 & 1 & 0 & 1 \\
    0 & 0 & 1 & 0 & 1 & 1 \\
    0 & 0 & 0 & 1 & 0 & 1 \\
    0 & 0 & 0 & 0 & 1 & 1
\end{pmatrix}
$$

Получаем:

\begin{equation}
    \begin{cases}
        x_5 = 1 \\
        x_4 = 1 \\
        x_3 \oplus x_5 = 1 \\
        x_2 \oplus x_3 \oplus x_4 = 1
    \end{cases} \implies \begin{cases}
        x_5 = 1 \\
        x_4 = 1 \\
        x_3 = 0 \\
        x_2 = 0 \\
        x_1 \oplus x_2 \oplus x_5 = 0 \implies x_1 = 1
    \end{cases}
\end{equation}

\textbf{Ответ:} $(10011)$

\paragraph{Пример №2}

\begin{equation}
    \begin{cases}
        (x_1 \downarrow x_3) \rightarrow x_2 = 1 \\
        (x_1 \oplus x_2) \sim x_3 = 1 \\
        (x_1 \lor \overline{x_2}) | x_3 = 0
    \end{cases}
\end{equation}

В данном случае мроще всего выполнить решение с помощью \textbf{таблицы истинности}:

$$\begin{pmatrix}
    x_1 & x_2 & x_3 & (1) & (2) & (3) & (4) & (5) & (6) \\
    0 & 0 & 0 & 1 & 0 & & & 1 &  & \\
    0 & 0 & 1 & 0 & 1 & 0 & 0 & 1 & \\
    0 & 1 & 0 & 1 & 0 & 1 & 0 & 0 & \\
    0 & 1 & 1 & 0 & 1 & 1 & 1 & 0 & 0 \\
    1 & 0 & 0 & 0 & 1 & 1 & 0 & 1 & 0\\
    1 & 0 & 1 & 0 & 1 & 1 & 1 & 1 & 0 \\
    1 & 1 & 0 & 0 & 1 & 0 & 1 & 1 & 0 \\
    1 & 1 & 1 & 0 & 1 & 0 & 0 & 1 & 0
\end{pmatrix}
$$

\textbf{Ответ:} $(110)$

\end{document}