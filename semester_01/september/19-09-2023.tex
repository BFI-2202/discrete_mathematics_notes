\documentclass{article}
\usepackage[utf8]{inputenc}

\usepackage[T2A]{fontenc}
\usepackage[utf8]{inputenc}
\usepackage[russian]{babel}

\usepackage{tabularx}
\usepackage{amsmath}
\usepackage{pgfplots}
\usepackage{geometry}
\usepackage{multicol}
\geometry{
    left=1cm,right=1cm,top=2cm,bottom=2cm
}
\newcommand*\diff{\mathop{}\!\mathrm{d}}

\newtheorem{definition}{Определение}
\newtheorem{theorem}{Теорема}

\DeclareMathOperator{\sign}{sign}

\usepackage{hyperref}
\hypersetup{
    colorlinks, citecolor=black, filecolor=black, linkcolor=black, urlcolor=black
}

\title{Дискретная математика}
\author{Лисид Лаконский}
\date{September 2023}

\begin{document}
\raggedright

\maketitle

\tableofcontents
\pagebreak

\section{Лекция — 19.09.2023}

\subsection{Теория множеств}

\begin{definition}
    Множество — совокупность каких-либо объектов.
\end{definition}

Обозначаются множества большими буквами. Например, в математике приняты следующие обозначениия числовых множеств: $N$ — натуральные, $Z$ — целые, $Q$ — рациональные, $R$ — действительные, $C$ — комплексные.

\subsubsection{Способы задания множеств}

\begin{enumerate}
    \item Непосредственным перечислением. Например: $A = \{ a_{1}, a_{2}, \dots, a_{n} \}$
    \item Условием, объединяющим его элементы. Например: $A = \{ \text{множество положительных чисел} \} = \{ a_{i}: \forall i a_{i} > 0 \}$
    \item Порождающей формулой или процедурой. Например: $B = \{ \text{множество положительных нечетных чисел} \} = \{ b_{i}: \forall i b_{i} = 2n-1, n \in N \}$
\end{enumerate}


\subsubsection{Типы множеств}

Множества делятся на 2 основных типа: \textbf{дискретный} и \textbf{непрерывный} тип.

\begin{definition}
    Между множествами A и B установлено взаимно однозначное соответствие $\Leftrightarrow$ $\forall a_i \in A: \exists b_{j} \in B$. То есть, $a_{i}$ соответствует $b_{j}$ и только ему.
\end{definition}

\begin{definition}
    Два множества $X$ и $Y$ называются \textbf{эквивалентными} или имеющими одинаковую мощность (обозначается $X \sim Y$), если между множествами $X$ и $Y$ может быть установлено взаимно однозначное соответствие
\end{definition}

\begin{definition}
    Множество называется \textbf{счетным}, если его элементы упорядочены и их можно пронумеровать, присвоить им порядковый номер.
\end{definition}

\begin{definition}
    Множество $M$ \textbf{дискретного типа}, если оно не более чем счётное, иначе оно \textbf{непрерывного типа}.
\end{definition}

\subsubsection{Операции над множествами}

\begin{enumerate}
    \item \textbf{объединение}, обозначается как $A \cup B$ — множество, содержащее все элементы из $A$ и $B$
    \item \textbf{разность}, обозначается как $A \setminus B$, реже $A - B$ — множество элементов $A$, не входящих в $B$
    \item \textbf{дополнение}, обозначается как $A \setminus A$ или $-A$ — множество всех элементов, не входящих в $A$ (в системах, использующих универсальное множество)
    \item \textbf{пересечение}, обозначается как $A \cap B$ — множество из элементов, содержащихся как в $A$, так и в $B$
    \item \textbf{симметрическая разность}, обозначается как  $A \bigtriangleup B$, $A - B$ — множество элементов, входящих только в одно из множеств — $A$ или $B$.
\end{enumerate}

\begin{definition}
    \textbf{Прямое, или декартово произведение} двух непустых множеств — множество, элементами которого являются все возможные упорядоченные пары элементов исходных множеств
\end{definition}

\paragraph{Свойства операций}

\begin{enumerate}
    \item Коммутативность \\
    $A \cup B = B \cup A$, $A \cap B = B \cap A$
    \item Ассоциативность \\
    $(A \cup B) \cup C = A \cup (B \cup C)$, $(A \cap B) \cap C = A \cap (B \cap C)$
    \item Дистрибутивность \\
    $A \cap (B \cup C) = (A \cap B) \cup (A \cap C)$, $A \cup (B \cap C) = (A \cup B) \cap (A \cup C)$
    \item Идемпотентность \\
    $A \cup A = A$, $A \cap A = A$
    \item Закон де Моргана \\
    $\overline{A \cup B} = \overline{A} \cap \overline{B}$, $\overline{A \cap B} = \overline{A} \cup \overline{B}$
    \item Операции с множествами \\
    $A \cup \emptyset = A$, $A \cap \emptyset = \emptyset$
    \item Операции с множеством \\
    $A \cup U = U \implies U = \overline{\emptyset} \implies \overline{U} = \emptyset$, $A \cap U = A$
    \item Законы поглощения \\
    $A \cup (A \cap B) = A$, $A \cup \overline{A} = U$, $A \cap (A \cup B) = A$, $A \cap \overline{A} = \emptyset$
\end{enumerate}

\subsubsection{Мера множеств}

\begin{enumerate}
    \item $|A \cup B| = |A| + |B| - |A \cap B|$
    \item $|A \cup B \cup C| = |A| + |B| + |C| - |A \cap B| - |A \cap C| - |B \cap C| + |A \cap B \cap C|$
    \item $|A \setminus B| = |A| - |A \cap B|$
\end{enumerate}

\subsubsection{Отношения множеств}

\textbf{Отношением} называется пара $(X, R)$, где $R \subseteq X \times X$. Так как элементами множества $X \times X$ являются упорядоченные пары, то отношение – это множество упорядоченных пар. Поскольку каждая пара связывает два элемента из $X$, то отношение называется \textbf{бинарным}. Тот факт, что два элемента $x$ и $y$ из $X$ связаны отношением $R$, обозначается $x R y$ или $(x, y) \in R$.

\begin{definition}
    \textbf{Областью определения} Бинарного отношения $R$ называется множество $\Delta_{R}$ элементов $x \in X$, для которых существуют такие $y \in Y$, что $x R y$.
\end{definition}

\begin{definition}
    \textbf{Областью значений} Бинарного отношения $R$ называется множество $E_{R}$ элементов $x \in X$, для которых существуют такие $y \in Y$, что $y R x$.
\end{definition}

\begin{definition}
    \textbf{Обратным отношением} $R^{-1}$ для бинарного отношения  $R$ называется множество упорядоченных пар $xRy$, таких, что $yRx$.
\end{definition}

\paragraph{Виды бинарных отношений}

\begin{enumerate}

    \item \textbf{Рефлексивное отношение} — двуместное отношение $R$, определённое на некотором множестве и отличающееся тем, что для любого $x$ этого множества элемент $x$ находится в отношении $R$ к самому себе, то есть для любого элемента $x$ этого множества имеет место $xRx$. Примеры рефлексивных отношений: равенство, одновременность, сходство.
    \item \textbf{Антирефлексивное отношение} (иррефлексивное отношение; так же, как антисимметричность не совпадает с несимметричностью, иррефлексивность не совпадает с нерефлексивностью) — бинарное отношение $R$, определённое на некотором множестве и отличающееся тем, что для любого элемента $x$ этого множества неверно, что оно находится в отношении $R$ к самому себе (неверно, что $xRx$).
    \item \textbf{Транзитивное отношение} — двуместное отношение $R$, определённое на некотором множестве и отличающееся тем, что для любых $x,y,z$ из $xRy$ и $yRz$ следует $x R z (xRy \wedge yRz\to xRz)$. Примеры транзитивных отношений: «больше», «меньше», «равно», «подобно», «выше», «севернее».
    \item \textbf{Нетранзитивное отношение} — двуместное отношение $R$, определённое на некотором множестве и отличающееся тем, что для любых $x,y,z$ этого множества из $xRy$ и $yRz$ не следует $xRz (\neg (xRy\wedge yRz\to xRz))$. Пример нетранзитивного отношения: «x отец y»
    \item \textbf{Симметричное отношение} — бинарное отношение $R$, определённое на некотором множестве и отличающееся тем, что для любых элементов $x$ и $y$ этого множества из того, что  $x$ находится к $y$ в отношении $R$, следует, что и $y$ находится в том же отношении к $x$ — $xRy\to yRx$. Примером симметричных отношений могут быть равенство, отношение эквивалентности, подобие, одновременность.
    \item \textbf{Антисимметричное отношение} — бинарное отношение $R$, определённое на некотором множестве и отличающееся тем, что для любых $x$ и $y$ из $xRy$ и $yRx$ следует $x=y$ (то есть $R и R^{{-1}}$ выполняются одновременно лишь для равных между собой членов).
    \item \textbf{Асимметричное отношение} — бинарное отношение $R$, определённое на некотором множестве и отличающееся тем, что для любых $x$ и $y$ из $xRy$ следует $\neg yRx$. Пример: отношения «больше» (>) и «меньше» (<).
    \item \textbf{Отношение эквивалентности} — бинарное отношение $R$ между объектами $x$ и $y$, являющееся одновременно рефлексивным, симметричным и транзитивным. Примеры: равенство, равномощность двух множеств, подобие, одновременность.
    \item \textbf{Отношение порядка} — отношение, обладающие только некоторыми из трёх свойств отношения эквивалентности: отношение рефлексивное и транзитивное, но несимметричное (например, «не больше») образует нестрогий порядок, а отношение транзитивное, но нерефлексивное и несимметричное (например, «меньше») — строгий порядок.
\end{enumerate}

\end{document}